\chapter{Conclusion}

In this chapter, we outline our experiences with used 3rd party libraries, elaborate on future work that can be done on the project and discuss our results.

\section{3rd party libraries}

Now we provide a quick summary of our experiences with the 3rd party libraries we decided to use.

\subsection{Cinder}
\subsection{Dear ImGui}
\subsection{CGAL}
\subsection{Assimp}
\subsection{Cereal}
\subsection{Threadpool}
\subsection{FreeType, FTGL and Poly2Tri}

\section{Future work}

In this section, we discuss the future work that could be done on this project. We divide the improvements that could be implemented into several categories:

\begin{itemize}
\item Improving existing core features
\item Adding \textit{quality of life} (QoL) changes to the GUI
\item Extending the toolset of the application
\end{itemize}

\subsection{Improving existing core features}

A few of Pepr3D's features and algorithms were developed by the team from scratch, since no solution satisfying our needs existed. These features are mainly the \textbf{Brush tool} and the \textbf{volumetric Export}. 

The brush tool uses computational geometry to subdivide triangles on the fly, which is not an easy task. Further work could be done by optimizing the brush tool to create better subdivisions and increasing the speed of the tool on bigger and more complex models. Our finished product is the best the team was able to come up with but with some more research, the tool can probably be optimized further.

The volumetric export (meaning the export which extrudes the faces inwards) is also a very complicated task, for which we have not found many solutions in any academic research or commercial products. We think that making this feature more robust would greatly improve the Pepr3D user experience.

\subsection{New quality of life features}

Since Pepr3D is a user-targeted application, the range of features the users have come to expect from the GUI of the program is vast. We implemented the basic subset of, what we think, are the most useful and important features -- such as hotkeys, tooltips and clear and simple user interface. However, there are many more features the users might benefit from, for example the radial menu around the mouse cursor, which we already discussed in Section \ref{sec:features}.

Other quality of life feature we got asked about by our colleagues during the development was a \textit{branching Undo \& Redo history}. This means that the command history would not be linear, but the user could go back a few commands from version B to version A, make new changes to version A, which would take him to version C. He could then compare versions B and C, which are both based on A and decide which he likes better.

The export GUI could also benefit from a semi-transparent overlay, that would show the user in real time, how the exported segments will look like and how deep they will be extruded. This could prevent some unwanted collisions or clipping behaviour.

\subsection{Extending the toolset}

When we designed the application's architecture, we put strong emphasis on allowing a potential developer to extend the toolset by adding other tools. We think we achieved this goal very well, because several of the tools require the same Geometry and Command API, which means we could add the tools and extend the functionality without implementing any additional functionality into Geometry or adding new Commands. This is the intended behaviour for the potential future developers.

If the new tool should require extending either the Geometry or Commands API, we strived to make the code educational -- if you need to create another command, you can read through one or two existing commands and then have a good understanding of how you should create your own.