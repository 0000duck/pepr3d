\chapter{Tools}

In this chapter, we explain the concepts behind each tool and the Geometry API the tools need.

A tool is the main programmable component which connects the low-level command structure we outlined above and the high-level UI components (such as color-pickers, the user performing brush strokes and file operations like exporting the file). Our design should allow for later advancements of the software -- adding a tool to the software should be a matter of writing the new tool's \texttt{Tool} class, unless the tool is advanced and needs some complicated custom geometry processing functions.

\medskip

Here we take a look at the \texttt{Tool} class overview:

\begin{lstlisting}
class Tool {
   public:
    Tool() = default;
    virtual ~Tool() = default;

    // UI properties
    virtual std::string getName() const = 0;
    virtual std::string getDescription() const;
    virtual std::optional<Hotkey> getHotkey(const Hotkeys& hotkeys) const;
    virtual std::string getIcon() const = 0;
    virtual bool isEnabled();

    // Action methods
    virtual void drawToSidePane(SidePane& sidePane){};
    virtual void drawToModelView(ModelView& modelView){};

	// Event methods (listed in the Table below)
	...

    virtual std::optional<std::size_t> safeIntersectMesh(MainApplication& mainApplication, const ci::Ray ray) final;

    virtual std::optional<DetailedTriangleId> safeIntersectDetailedMesh(MainApplication& mainApplication, const ci::Ray ray) final;
};
\end{lstlisting}

As we can see, it is a simple abstract interface class. There are some \textit{UI properties} like \texttt{getName} and \texttt{getDescription}, as well as some rendering methods like \texttt{drawToSidePane}. It contains a few \texttt{final} methods, which are just a exception-safe methods to call on the \texttt{Geometry} class, implemented in the interface, because every tool needs them. The also are multiple \textit{Event methods}, which we have omitted from the code here and will examine further in the following table.

\begin{table}[H]
\centering
\begin{tabular}{ll}
\textbf{Name}         & \textbf{Invoked if ...}                       \\
\texttt{onModelViewMouseDown}  & mouse gets pressed down in the model view     \\
\texttt{onModelViewMouseDrag}  & mouse gets dragged in the model view          \\
\texttt{onModelViewMouseUp}    & mouse button gets let go of in the model view \\
\texttt{onModelViewMouseWheel} & mouse wheel is scrolled in the model view     \\
\texttt{onModelViewMouseMove}  & mouse is moved in the model view              \\
\texttt{onToolSelect}          & this tool is selected                         \\
\texttt{onToolDeselect}        & this tool is deselected                       \\
\texttt{onNewGeometryLoaded}   & new model is loaded (imported or opened)
\end{tabular}
\label{tab:events}
\caption{The overview of different events available to any \texttt{Tool} class.}
\end{table}

When implementing a new tool, not all of these events have to be specified. The tool only listening and can act accordingly if it needs to.

The tool's \textbf{user interface} is implemented in the \texttt{drawToSidePane} method. During the rendering, if the tool is active, this method gets called to fill the SidePane with ImGui widgets, which allow the user to alter the tool's properties.

We will not discuss each and every tool we have implemented, we will, however, mention a few things that most of the tools have in common.

\section{Specific tool details}

Most, if not all of the tools we implemented have to access the \texttt{Geometry} and \texttt{CommandManager} classes. Since the \texttt{MainApplication} object holds both of these, the tool has to get a reference to the parent \texttt{MainApplication} object. This is why most of our tools have this piece of code in common:

\begin{lstlisting}
public:
	explicit PaintBucket(MainApplication& app) : mApplication(app) {}
private:
	MainApplication& mApplication;
\end{lstlisting}

The tool then accesses the \texttt{CommandManager} through \texttt{mApplication}'s method \texttt{getCommandManager()} and the current geometry data through the \texttt{getCurrentGeometry()} method.