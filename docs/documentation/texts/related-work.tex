
\chapter{Related work}

Based on our own research and the analysis of the experts from Prusa Research s.r.o, there, at the moment, does not exist a software that does what this project is trying to achieve. Here we present a simple list of software that could be used to achieve the same results as our program delivers. We highlight the pros and cons of each program to show that our goal is sufficiently unique.

\section{Autodesk Meshmixer}

The closest existing software is Autodesk Meshmixer \footnote{http://www.meshmixer.com/}, which is very complicated and is not targeted for FDM printing specifically. As such, it includes a lot of features that are not important for the FDM users and end up being confusing.

\section{Microsoft 3D Builder}

Microsoft 3D Builder \footnote{https://www.microsoft.com/en-us/p/3d-builder/9wzdncrfj3t6?activetab=pivot\%3Aoverviewtab} is another application that handles 3D models but we have not found a way to make it create anything remotely applicable to FDM printing.

\section{Complex 3D editors}

Any 3D computer graphics program designed to handle 3D models which allows for the model to be created or split by colors manually. This section would include software as 3ds Max, Maya or Cinema4D. Using these applications, however, would be very time comsuming for the user and practically unusable on a larger scale.

