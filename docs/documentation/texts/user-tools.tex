\chapter{Tools}

This chapter covers all the tools the user has at his disposal. We explain each tool's purpose and all the parameters the user can set.

\section{Triangle Painter}

Triangle Painter is the simplest tool of Pepr3D. It allows the user to color a single triangle with a selected color. This can be performed either by a single click on the model's triangle or by dragging the mouse over several triangles with the left mouse button pressed down.

The triangle that is currently hovered (has the mouse cursor over it) will be highlighted on the model's surface with a different border color.

The only property the user is able to select in this tool is the current color from the color palette.

Pressing the \textit{Undo} will undo the last stroke of the triangle painter. This means it will undo \textbf{the whole} stroke, if the user dragged the mouse over several triangles.

\section{Bucket Painter}

Bucket Painter is a simple tool that can be used to achieve sophisticated results easily. This tool works as one is used to from image editing software like GIMP \footnote{https://www.gimp.org/} or Adobe Photoshop \footnote{https://www.adobe.com/products/photoshop.html} -- it starts colouring every triangle it can reach, starting with the triangle the user clicked on.

\subsection{Properties}

The properties of this tool revolve around the spread of the bucket. This is something we call \textit{stopping criteria}. We now list all properties of the tool and explain each one in detail.

\begin{itemize}
\item \textbf{Color Palette} -- This widget allows the user to select the current active color. The selected color will be spread by the bucket. Customizing the palette can be performed in the \textit{Settings} panel.

\item \textbf{Paint while dragging} -- \textit{On / Off} -- This checkbox specifies whether the bucket painter will only function by clicking on single triangles (\textit{Off}) or will bucket spread continuously if the user drags the mouse in a stroke (\textit{On}). We recommend leaving this \textit{On} unless it disrupts you or the model you are working on is very big.

\item \textbf{Color whole model} -- \textit{On / Off} -- We have mentioned \textit{stopping criteria} in the beginning. This is the  first choice the user can make that affects the stop of the bucket spread. If the user selects this option, the Bucket Painter will simply color the whole region of the model. If the model is a single mesh, it will color the whole model. Turning this \textit{On} will hide the other options. Turning this \textit{Off} allows the user to specify the \textit{stopping criteria}.

\item \textbf{Stop on different color} -- \textit{On / Off} -- The simplest \textit{stopping criterion}. The spread will only re-paint triangles which have the same color as the triangle the user clicked on. Additionally, the spread will stop if a new color is met. If this is the only criterion that is enabled, the Bucket Painter will work exactly as we are used to from image editors. This is the default setting of the tool.

\item \textbf{Stop on sharp edges} -- \textit{On / Off} -- A second \textit{stopping criterion} which can be enabled or disabled. Enabling it expands the user interface to allow the user to modify the criterion. This criterion will not care about the color the user clicked on, and only stops from spreading to the neighbouring triangle, if the neighbouring triangle is at a greater angle than specified. The exact behaviour is specified by the following properties.

\item \textbf{Maximum angle} -- \textit{$0\degree$ -- $180\degree$} -- Specifies the angle which the two neighbouring triangles have to be angled at for the bucket painter to stop spreading. If the angle between the two triangles is greater than this value, the spread will not color the triangle and will stop.

\item \textbf{Angles to compare} -- \textit{With starting triangle / Neighbouring triangles} -- the last choice in the sharp edges \textit{stopping criterion}. If the user selects \textit{With starting triangle}, the angle will be measured between the triangle the user clicked on and the triangle currently being coloured. For example, if this option is chosen, the angle is set to $95\degree$ and a single face of the cube is clicked, all faces of the cube except the opposite one are coloured. This is because the opposite face is at an $180\degree$ angle. If the user selects \textit{Neighbouring triangles} and uses the $95\degree$ setting again, the whole cube will get coloured, because there are no faces on the cube that are at an angle greater than $95\degree$.

\end{itemize}

Both of the \textit{stopping criteria} can be selected together. The spread stops when one of the criteria is not fulfilled -- both of the criteria must be fulfilled for the spread to continue.

\section{Automatic Segmentation}

Automatic segmentation is a powerful tool which allows the user to quickly achieve the baseline colouring of the model, which then can be detailed to the user's liking. This is achieved by separating the model into several segments based on the thickness. These segments then can be quickly coloured individually. The user can select the sensitivity of this segmentation, which allows him to control the level of detail (for example, low sensitivity might only segment the body and limbs of a character, whilst high sensitivity will also segment the fingers, ears and horns).

\subsection{Properties}

\begin{itemize}

\item \textbf{Compute SDF values} -- \textit{button} -- before anything can be done in this tool, the user is asked to compute the SDF values. This is the data that is required to perform a successful segmentation. This computation might take a long time to perform, depending on your model size. For low-poly models (e.g. 1000 triangles), this computation is instantaneous. If you already performed the computation in a different tool (like export or the other segmentation), this option will not be visible.

\item \textbf{Segment!} -- \textit{button} -- this button starts the segmentation process. The default values are set so the segmentation returns viable results in most cases. If you did not set any of the following properties and the segmentation returned an undesirable number of segments (like only one or too many), modify the following properties.

\item \textbf{Robustness} -- $0\%$ -- $100\%$ -- a magic parameter. The meaning of this parameter is somewhat obscured but the best way to imagine this setting is the quality or robustness of the segmentation. Higher values take longer to compute but might give better results -- a higher value might merge two segments that are somewhat related, which the lower values will not recognize. The default setting is the team's best effort to balance the performance and quality.

\item \textbf{Edge tolerance} -- $0\%$ -- $100\%$ -- This parameter specifies how the algorithm should understand "thickness". If you set this value very high, the algorithm will tend to merge more segments together, resulting in a lower amount of segments. If you set this very low, every nook and crease will signal the algorithm to create a new segment, thus resulting in a higher amount of segments. This is the \textbf{primary means} to control this tool and we recommend adjusting this slider over the previous one to change the main behaviour.

\item \textbf{Color Palette} -- same as in the previous tools, this widget allows the user to select a color to assign to each segment.


\end{itemize}

\subsection{Segmentation}

To adjust the segmentation, we recommend first trying to adjust the \textit{Edge tolerance} slider, and only after experimenting with this slider to change the \textit{Robustness}.

After these settings are adjusted and the user clicks on the \textit{Segmentation} button, a list of segments will appear, along with the number of segments created. The user is then instructed to assign a color from the color palette to each segment. This can be done in two ways -- either by clicking directly on the model or by clicking on the "Segment \#" button. After clicking on one of these two regions, the color selected in to color palette will get assigned to the segment.

After \textbf{all segments} have been assigned a color from the color palette, the user is able to click \textit{Accept} to color the model this way. Should the user be dissatisfied with the colouring, he can either click \textit{Cancel} or \textit{Segment!} again, to completely undo the whole segmentation and start from scratch.


\section{Manual Segmentation}

Manual segmentation is a similar tool to the Automatic segmentation we discussed in the last section. The difference in these two tools is that while Automatic segmentation is a very global tool (since it segments the whole model at once), Manual segmentation is designed to be local. The user can color a handful of triangles with a single color and then manually adjust the spread of this color over the segment the triangles define. This description is rather abstract, but hopefully it will get clearer once we discuss the properties.


\subsection{Properties}

\begin{itemize}

\item \textbf{Compute SDF values} -- \textit{button} -- before anything can be done in this tool, the user is asked to compute the SDF values. This is the data that is required to perform a successful segmentation. This computation might take a long time to perform, depending on your model size. For low-poly models (e.g. 1000 triangles), this computation is instantaneous. If you already performed the computation in a different tool (like export or the other segmentation), this option will not be visible.

\item \textbf{Color palette} -- once SDF values of the object have been computed, the user is presented with a sidepane very similar to the Triangle painter tool. This widget allows the user to select the current color. The color will be used while initializing the segments on the model.

\item \textbf{Spread} -- $0\%$ -- $100\%$ -- once a single triangle is coloured on the model, additional options appear. One of them is the \textit{Spread} slider. This slider is analogous to the \textit{Edge tolerance} slider in Automatic segmentation, since it controls how much each coloured triangle will spread its color among its local neighbourhood. If the spread is $0\%$, only the triangle is coloured. Increasing it to $100\%$ will color all triangles of the model (unless a second segment is competing).

\item \textbf{Hard edges} -- \textit{On / Off} -- if this option is turned on, the spread of one color will stop upon meeting a second one's border and will not attempt to color any other triangles. Use this option if the segments you are colouring are well defined and differ in thickness a lot.

\item \textbf{Region overlap} -- \textit{On / Off} --once the user turns this option on, the spreading regions will overlap freely and the last color (rightmost) will always win if \textit{Spread} is turned to $100\%$. Use this option if the borders of the segments converged a little soon or late and you would like one color to expand a little more. This option is not good if the spread is turned to high percentages, since a lot of the segments will overlap and information will be lost.

\end{itemize}

\subsection{Segmentation behaviour}

If no checkboxes are turned on, the color spreads to the global optimum segmentation. This means that if you set the \textit{Spread} to a high percentage, there might be several discontinuous segments coloured by the same color, because that is globally optimal to the input. This setting is a good starting point, since it does not restrict the spread too much, but does not allow it to roam unlimited as well. Use the other options to tweak the spread after you understand how this model will get segmented.

The spreading algorithm uses the SDF function (thickness of the model) to calculate segments. This means that this tool is not able to grow segments on a model which does not vary in thickness. It is also very important to be aware of the fact that several non-neighbouring parts of the model might have the same thickness (like ears, hand fingers and feet fingers) and thus the global optimum colors all of these with the same color (use \textit{Hard edges} to counter this behaviour).

After the user is satisfied with the spread, clicking \textit{Apply} will confirm this re-colouring. Clicking cancel will return the model into the state before Manual segmentation started.

As a last tip, we recommend using a single-triangle strokes to initialize the segments at the beginning. This is using many triangles accelerates the spread a lot, which ultimately gives the user less fidelity and time to see what is happening.

We want to stress that this tool is an advanced tool, which should be used to fine-tune an already coloured model's details, not as a means to primarily color the model upon importing it (Automatic segmentation is a lot better tool in that case).